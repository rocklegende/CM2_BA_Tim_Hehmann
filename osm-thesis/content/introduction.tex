\chapter{Introduction}

In order to protect critical areas from unauthorized access, most office buildings use an access control system that grants or denies access to gates and doors based on the permissions of the employee.

The most common way to authenticate in these systems is either by knowledge (keypad with pin login) or ownership (NFC chipcard).
But not only office buildings, also gyms, public transportation services and universities use access control systems with the same ways to authenticate. This results in a lot of different cards and passwords for the user. The management of these can easily be overwhelming and once a thief obtained one of these there is a possibility for an attack.

\section{Context of the project}
\label{Context of the project}

The bachelors project from 2016 'Passwords Are Obsolete - User Authentication Using Wearables And Mobile Devices' tried to solve this problem and came up with \emph{BAuth}\footnote{\url{https://play.google.com/store/apps/details?id=com.nexenio.behaviourauthentication&hl=de}} (short for Behavioural Authentication), an app that makes it possible to authenticate the user solely on his behaviour. This is done by continously analysing the sensor data from smartphone/-watch and calculating a \emph{trustlevel}, a value that determines how certain it is, that the device is in possession of the correct owner.

This new way of authenticating solves the management issue of cards and passwords by authenticating directly with the device. It also lowers the security risk in an event of a theft, because reading a \emph{wrong} behaviour just for a few meters results in a significant drop in trustlevel, thus denying access almost immediately.

But due to the fact that existing access management solutions don't work with this authentication method, the desired protection of certain areas is left open. This is where the scope of this years bachelors project started. 

The goal of our project was to create an access management platform that is suited to work with BAuth. The facility management and also companies should be able to define which employees can access which gates. It should also be possible to set the minimal trustlevel that is needed to enter or leave a certain gate, which enables to define which rooms need more protection than others.

With this solution, BAuth could be used in a real world scenario.

\section{Context of this thesis}
\label{Context of this thesis}

The management of an office building with multiple floors and multiple gates can be a challenging task for the facility management team. To prevent loosing the overview of the facility, the usage of an interactive floorplan can be helpful. The implementation of such a plan was also part of our projects scope.

In our access management system this graphical plan gives insight about the different gates in the building, including the access decisions made at these and information about the person that tried to access. Furthermore it visualises how many persons are approximately in a room and at which gates an alarm occured. 

This information could be used by the facility management team to see how heavily the gates are used, where a possible security threat exists and also if a room is currently at risk of not being evacuated safely. 
In general it improves the overall view of the facility and could lead to a faster use of our other access management tools.

This bachelor thesis will compare different approaches for implementing such a floorplan and present the chosen approach for this project. To accomplish this it will be guided by the following structure:

In the second chapter, related work gets discussed. This will showcase the different technologies that are available right now for creating an interactive floorplan, including the strengths and weaknesses for each one. In this chapter we will also give a short introduction to the tools and formats we will mention in the further chapters.

The third chapter describes our chosen approach and the architecture and components behind it.

In the fourth chapter we will present our implementation for a live indoor floorplan. This will show the solutions used for logging gate events and displaying a real-time floorplan with an integrated heatmap.

The topic of the fifth chapter is evaluation. In this chapter the performance of a live plan in differently sized simulation environments gets analysed. Furthermore it will discuss the protection of privacy in a live plan, especially focussing on the personal informations that get shown in the floorplan.

The sixth chapter will present what further features could be implemented in the future and what needs to be done before actually deploying it into production. 

The seventh and last chapter will wrap the thesis up.