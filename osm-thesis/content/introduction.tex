\chapter{Introduction}

In order to protect critical areas from unauthorized access, most office buildings use an access control system that grants/denies access to gates and doors based on the permission of the employee.

The most common authentication methods in these systems are RFID/NFC keycards/chips or password logins.
But not only office buildings, also gyms, public transportation and universities use access control systems with the same methods. This results in a lot of different cards and passwords for the user. The management of these can easily be overwhelming and once a thief obtained one of these there is a possibility for an attack.

\section{Context of the project}
\label{Context of the project}

The bachelors project from 2016 'Passwords Are Obsolete - User Authentication Using Wearables And Mobile Devices' tried to solve this problem and came up with \emph{BAuth}\footnote{\url{https://play.google.com/store/apps/details?id=com.nexenio.behaviourauthentication&hl=de}} (short for Behavioural Authentication), an app that makes it possible to authenticate the user solely on his behaviour. This is done by continously analysing the sensor data from smartphone/-watch and calculating a \emph{trustlevel}, a value that determines how certain it is, that the device is in possession of the correct owner.

This new way of authenticating solves the management issue of cards and passwords by authenticating directly with the device. It also lowers the security risk in an event of a theft, because reading a \emph{wrong} behaviour just for a few meters results in a significant drop of the trustlevel, thus denying access almost immediately.

But due to the fact that existing access management solutions don't work with this authentication method, the desired protection of certain areas is left open. This is where the scope of this years bachelors project started. 

The goal of this project was to create an access management platform that is suited to work with BAuth. The facility management and also companies should be able to define which employees can access which gates. It should also be possible to set the minimal trustlevel that is needed to enter or leave a certain gate.

With this solution, BAuth could be used in a real world scenario.

\section{Context of the thesis}
\label{Context of the thesis}

Another part of the scope was to create an \emph{interactive floorplan}. This graphical plan gives insight about the different gates in the building, including the access decisions made at these. This information could be used by the facility management team to see how heavily the gates are used, where a possible security thread occured and what gates are malfunctioning. 

But also employees could have uses for this plan. Another feature in the scope of this project was the ability for employees to find each other on the floorplan by sharing their current position. This could lead to faster communication between employees and increase productivity.

This bachelor thesis will compare different approaches for implementing such a floorplan and present the chosen approach for this project. To accomplish this it will follow a certain structure. 

In the second chapter, related work gets discussed. This will showcase the different technologies that are available right now for creating an interactive floorplan, including the strengths and weaknesses for each one.

The third chapter describes our chosen approach and the architecture and components behind it.

The topic of the fourth chapter is evaluation. In this chapter the performance of a live plan in differently sized simulation environments gets analysed. Furthermore it will discuss the protection of privacy in a live plan, especially focussing on the position-sharing feature mentioned above.

The fifth and last chapter will present what further features could be implemented in the future and what needs to be done before actually deploying it into production. Finally the thesis gets wrapped up.