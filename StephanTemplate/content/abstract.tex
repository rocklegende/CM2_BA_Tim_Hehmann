% => Wenn die Arbeit auf Deutsch verfasst wurde, verlangt das Studienreferat KEINEN englischen Abstract

% % englischer Abstract
\null\vfil
\begin{otherlanguage}{english}
\begin{center}\textsf{\textbf{\abstractname}}\end{center}

\noindent Guaranteering the safety in facilities is one of the main tasks of the facility management. This includes ensuring a safe evacuation in case of an alarm, the quick reaction to possible threats and also the protection of the doors and gates. Especially with the deployment of \emph{Behavioural Authentication} the safety level for each of these access points can be finely set. In large office buildings this management and the guarantee of the safety can only be done with the help of visual aids.

This thesis presents an implementation of one possible visual aid: the interactive floorplan. Furthermore it will showcase the tools that are available for creating such a plan and also evaluate how good it performs indifferent simulation environments.

\end{otherlanguage}
\vfil\null


% => Wenn die Arbeit auf Englisch verfasst wurde, verlangt das Studienreferat einen englischen UND deutschen Abstract (der dt. Abstract kann dann ggf. auch ans Ende der Arbeit)

% deutsche Zusammenfassung
\null\vfil
\begin{otherlanguage}{ngerman}
\begin{center}\textsf{\textbf{\abstractname}}\end{center}

\noindent Die Sicherheit in Gebäuden zu gewährleisten gehört zu einer der Kernaufgaben der Gebäudeverwaltung. Dazu zählt die Gewährleistung einer Evakuierung im Notfall, die schnelle Reaktion auf mögliche Gefahren, aber auch die Absicherung der einzelnen Türen. Bei letzterem ist besonders mit dem Einsatz von verhaltensbasierter Authentifizierung es möglich, feingranulare Sicherheitsstufen für die einzelnen Türen festzulegen. Bei großen Bürokomplexen kann diese Verwaltung und Sicherheitsgewährleistung nur mit visuellen Mitteln bewältigt werden. 

Diese Arbeit präsentiert eine Umsetzung einer dieser Mittel: den interaktiven Gebäudeplan. Dabei wird darauf eingegangen mit welchen Werkzeugen ein solcher Plan implementiert werden kann und auch gleichzeitig evaluiert, wie performant dieser ist in  verschiedenen Simulationsumgebungen.

\end{otherlanguage}
\vfil\null



