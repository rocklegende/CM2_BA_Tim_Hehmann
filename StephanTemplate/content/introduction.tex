\section{Introduction}

To protect critical areas from unauthorized access, most office buildings use an access control system that grants or denies access to gates and doors based on the permissions of the employee.

The most common way to authenticate in these systems is either by knowledge (keypad with pin login) or ownership (NFC chip card).
Not only office buildings, but also gyms, public transportation services, and universities use access control systems with the same ways to authenticate. This results in a lot of different cards and passwords for the user. The management of these can easily be overwhelming and once a thief obtained one of these there is a possibility for an attack.

\subsection{Context of the project}
\label{Context of the project}

The bachelor's project from 2016 'Passwords Are Obsolete - User Authentication Using Wearables And Mobile Devices' tried to solve this problem by building an app that makes it possible to authenticate the user solely on his behavior. This is done by continuously analyzing the sensor data from smartphone/-watch and calculating a \emph{trustlevel}, a value that determines how certain it is, that the device is in possession of the correct owner.
Together with our project partner \emph{neXenio}\footnote{\url{https://www.nexenio.com/}}, some members of the bachelors project develop this app continuously further by the name of \emph{BAuth}\footnote{\url{https://play.google.com/store/apps/details?id=com.nexenio.behaviourauthentication&hl=de}}, 

Besides authenticating the user by behavior, the smartphone (with BAuth installed) is also already able to communicate with the gates via Bluetooth. But the management of the access rights of each employee for every single gate is currently done by hardcoding a list of authorized employees directly on to the hardware inside the gates. Although there are a lot of access management systems out there that provide automation for the management of access rights, none of them fits the needs for the app of our predecessors. Additionally, these systems are all closed source and cannot be extended to also work for this use case. 

The goal of our project was to create an access management platform that is suited to work with BAuth and makes access management more comfortable. The facility management and also companies should be able to define which employees can access which gates. It should also be possible to set the minimum trust level that is needed to enter or leave a certain gate, which enables them to define which rooms need more protection than others.

With this solution, BAuth could be used in a real-world scenario.

\subsection{Context of this thesis}
\label{Context of this thesis}


The management of an office building with multiple floors and multiple gates can be a challenging task for the facility management team. To prevent losing the overview of the facility, the usage of an interactive floorplan can be helpful. The implementation of such a plan was also part of the scope of our project and forms the main topic of this thesis.

In our access management system, this graphical plan gives insight about the different gates in the building, including the access decisions made at these and information about the person that tried to access. Furthermore, it visualizes how many persons are approximately in a room and at which gates an alarm occurred. 

This information could be used by the facility management team to see how heavily the gates are used, where a possible security threat exists and also if a room is currently at risk of not being evacuated safely. 
In general, it improves the overall view of the facility and could lead to faster use of our other access management tools.

This bachelor thesis will showcase an approach for the design and implementation of such a floorplan. To accomplish this it will be guided by the following structure:

In the second chapter, related work gets discussed. This will showcase different solutions from software companies that already offer services for similar requirements.

The third chapter describes our chosen approach and the architecture and components behind it. Besides that, we will present the requirements that we received for the implementation of the interactive floorplan and what external input we have at disposal.

In the fourth chapter, we will present our implementation for an interactive floorplan. This will present the solutions used to fulfill the requirements explained in the Concept chapter.

The topic of the fifth chapter is evaluation. In this chapter, we will evaluate how well our floorplan fulfills its requirements. Furthermore, it will discuss the protection of data privacy in a floorplan, especially focussing on the personal information that gets shown in the floorplan.

The sixth chapter will present what further features could be implemented in the future and what needs to be done before actually deploying it into production. 

The seventh and last chapter will wrap the thesis up.

\clearpage