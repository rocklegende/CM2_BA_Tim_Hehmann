\section{Future Works}

This work leaves a lot of tasks open before the deployment into a production environment would be possible.

The most important feature is the functionality to link gates together, thereby creating a gate graph. Currently, gates are directly connected to one room and define the number of people for that room. But in an office where multiple gates are installed, some gates can only be reached by passing another gate first. The current version of the implemented floorplan would lead to wrong calculations for the number of people in the rooms if situation occurs: An entry event through a gate \(g_2\) that is behind another gate \(g_1\) would add one person to the room connected to \(g_2\) but not subtract the person from the room connected to \(g_1\) where the person came from. To implement this, we would need further information from the user about the predecessors and successors of each gate.

Further, the current implementation only supports connecting a gate to exactly one room, but of course multiple rooms could be protected by a single gate. Therefore the user should have the ability to connect multiplerooms to a gate.

Office buildings usually have more than one floor. But the implemented floorplan is not designed to work with multiple floors yet. This could be implemented through the already mentioned indoor plugin for Leaflet, which features the option to handle GeoJSON files with integrated data about the floorlevel of each feature and included map controls to switch between the different floors. We would then also need to persist the floorlevel for each marker, so we can add them to the correct floors.

This also leads to another usability issue: the manual positioning of the gates. A better solution would be to integrate with already existing GeoJSON where gates are also included as features.

The next improvement would be the option for the facility manager to set the maximum limit of persons in a room themselves or define a person per squaremeter limit (which would require to know or calculate the  squaremeters of each room). Currently this maximum value is hardcoded and therefore is the same for large and small rooms.

Every time an event happens, a request is send to the backend for the updated number of people that are behind the gate where the event occured. This results in a lot of traffic at each notification about an event. By only requesting the number of people for each gate at startup and the client updating the occupancy state by itself (simply adding to or subtracting from the  amount of people in the room) this traffic could be reduced and improve the performance of the floorplan. 

Some of the work presented in the Related Work chapter include the option to review and replay events from the past. This could also be useful for our use case. Through this option the facility manager could view how the heatmap developed over time and also replay missed events, giving them insights about which time intervals and areas are critical for the safety of the employees.

\clearpage

