\usepackage{array}
\usepackage{supertabular}
\usepackage{colortbl}

% TODO
\newcommandx{\unsure}[2][1=]{\todo[linecolor=red,backgroundcolor=red!25,bordercolor=red,#1]{#2}}
\newcommandx{\change}[2][1=]{\todo[linecolor=blue,backgroundcolor=blue!25,bordercolor=blue,#1]{#2}}
\newcommandx{\info}[2][1=]{\todo[linecolor=green,backgroundcolor=green!25,bordercolor=green,#1]{#2}}
\newcommandx{\improvement}[2][1=]{\todo[linecolor=green,backgroundcolor=green!25,bordercolor=green,#1]{#2}}
\newcommandx{\thiswillnotshow}[2][1=]{\todo[disable,#1]{#2}}

\newcommand{\frontmatter}{\pagenumbering{roman}}
\newcommand{\mainmatter}{\pagenumbering{arabic}\setcounter{page}{1}}

\newcounter{todocounter}
\setcounter{todocounter}{0}
\newcounter{authcounter}
\setcounter{authcounter}{0}
%%% BEGIN Write TODO in File
\def\getdefhelp#1->#2\endhelp{#2}
\def\getdef#1#2{\edef#2{\expandafter\getdefhelp\meaning#1\endhelp}}
\newwrite\TodoDatei
\newwrite\AuthorDatei
\openout\AuthorDatei=author.out
\newcommand{\WriteTodo}[2]{%
	\def\Cont{#2}
	\getdef\Cont\Content
	\edef\WriteIndex{%
		\write\TodoDatei{\string\textcolor{#1}{\string\textbf{\Content}}\string\dotfill\string\pageref{todo:\thetodocounter}}}%
	\WriteIndex}

\newcommand{\WriteAuthor}[1]{%
	\def\Cont{#1}
	\getdef\Cont\Content
	\edef\WriteAuth{\write\AuthorDatei{\Content\string\dotfill\string\ref{auth:\theauthcounter}}}
	\WriteAuth}


%%% END Write TODO in File

% command \BibTeX
\def\BibTeX{{\rm B\kern-.05em{\sc i\kern-.025em b}\kern-.08em
     T\kern-.1667em\lower.7ex\hbox{E}\kern-.125emX}} 

% day in journal: \journalday{date}{titel}{persons}{aktivity}
\newcommand{\journalday}[4]{%
\def\titleTmp{#2}
\subsection*{#1\ifx\titleTmp\empty{}\else{: #2}\fi}
\begin{center}
\begin{tabularx}{\textwidth}{@{}lX@{}}
	Anwesende: & #3\\
	Vorgang: & #4
\end{tabularx}
\end{center}
}

% errorreport: \errorreport{date}{error}{reason}{solution}
\newcommand{\errorreport}[4]{%
\def\dateTmp{#1}
\def\errorTmp{#2}
\def\reasonTmp{#3}
\def\solutionTmp{#4}
\ifx\errorTmp\empty{}\else{%
\subsection*{\ifx\dateTmp\empty{}\else{\hfill(#1)\\}\fi Problem: #2}%
{\begin{center}%
\vskip-1ex%
\begin{tabularx}{\linewidth}{@{}lX@{}}
	Ursache: & \ifx\reasonTmp\empty{unbekannt}\else{#3}\fi\\
	L�sung: & \ifx\solutionTmp\empty{unbekannt}\else{#4}\fi\\
\end{tabularx}%
\end{center}%
}}\fi}

% code: \code[textcolor]{backgroundcolor}{content}
\newcommand{\code}[3][black]{%
	\begin{flushleft}	
		\ttfamily
		\small
		\fcolorbox{#1}{#2}{\textcolor{#1}{\shortstack[l]{#3}}}
	\end{flushleft}
}

% code with white borderline
\newcommand{\codeblank}[3][black]{%
	\begin{flushleft}	
		\ttfamily
		\small
		\fcolorbox{white}{#2}{\textcolor{#1}{\shortstack[l]{#3}}}
	\end{flushleft}
}

% centered code: \centercode[textcolor]{backgroundcolor}{content}
\newcommand{\centercode}[3][black]{%
	\begin{center}	
		\ttfamily
		\small
		\fcolorbox{#1}{#2}{\textcolor{#1}{\shortstack[l]{#3}}}
	\end{center}
}

% centered code with white borderline
\newcommand{\centercodeblank}[3][black]{%
	\begin{center}	
		\ttfamily
		\small
		\fcolorbox{white}{#2}{\textcolor{#1}{\shortstack[l]{#3}}}
	\end{center}
}


% annotation in colored box: \annot[text- and bordercolor]{backgroudcolor}{contents}
\newcommand{\annot}[3][black]{%
	\begin{center}	
		\fcolorbox{#1}{#2}{\textcolor{#1}{\shortstack[l]{\vspace*{1ex}\\\hspace*{.025\textwidth}\textbf{Anmerkung:}\\\hspace*{.05\textwidth}\parbox{.88\textwidth}{#3\vspace*{2ex}}\hspace*{.05\textwidth}}}}
	\end{center}
}

% annotation in colored box: \annot[text- and bordercolor]{backgroudcolor}{contents}
\newcommand{\hint}[3][black]{%
\begin{figure}[!t]
  \centering
  \fcolorbox{#1}{#2}{
    \begin{minipage}{.96\linewidth}
      \hspace*{.025\linewidth}\parbox{.93\linewidth}{\textbf{Hinweis:}}\hspace*{.025\linewidth}\\
      \hspace*{.05\linewidth }\parbox{.88\linewidth}{\vspace*{3ex}#3\vspace*{3ex}}\hspace*{.05\linewidth}
    \end{minipage}
  }
\end{figure}
}

\newcommand{\colorparbox}[3][.985\textwidth]{%
\begin{flushleft}
\fcolorbox{black}{#2}{\parbox{#1}{#3}}
\end{flushleft}
}

\newcounter{versionID}
\newenvironment{versioning}[1][hpiblue4]{%
\setcounter{versionID}{0}
\begin{center}	
	\tablefirsthead{%
		\hline
		\rowcolor{#1}
		\parbox[c][2em][c]{\linewidth}{\centering\textbf{lfd. Nr.}} &
		\parbox[c][2em][c]{\linewidth}{\centering\textbf{Bearbeiter}} & 
		\parbox[c][2em][c]{\linewidth}{\centering\textbf{�nderungen}}\\
		\hline}
	\tablehead{%
		\extrahead
		\hline
		\rowcolor{#1}
		\parbox[c][2em][c]{\linewidth}{\centering\textbf{lfd. Nr.}} &
		\parbox[c][2em][c]{\linewidth}{\centering\textbf{Bearbeiter}} & 
		\parbox[c][2em][c]{\linewidth}{\centering\textbf{�nderungen}}\\
		\hline}
	\tabletail{%
		\hline
		\multicolumn{3}{|r|}{\cellcolor{hpiblue4}\small\sl Fortsetzung auf der n�chsten Seite}\\
		\hline}
	\tablelasttail{}
	\begin{supertabular}{|p{.1\linewidth}|p{.25\linewidth}|p{.5\linewidth}|}
	}{%
	\end{supertabular}
\end{center}
\newwrite\VersionDatei
\openout\VersionDatei=theversion.aux
\write\VersionDatei{\theversionID}
\closeout\VersionDatei
%\vfill
}

\newcommand{\version}[2]{%
\parbox{\linewidth}{\centering\stepcounter{versionID}\theversionID} & 
\parbox[t]{\linewidth}{\centering#1} &
#2 \\\hline
}

\newcommand{\currentversion}[1][]{
\def\test{#1}
\def\drafttest{draft}
\def\finaltest{final}
\ifx\test\drafttest
	\def\versiontext{(Entwurf)}
\else
	\ifx\test\finaltest
		\def\versiontext{(Final)}
	\else
		\def\versiontext{}
	\fi
\fi
\vskip.3cm
\newread\DatenDatei
\openin\DatenDatei=theversion.aux
\ifeof\DatenDatei\def\curVers{---}\else\read\DatenDatei to \curVers\fi
\closein\DatenDatei
{\small Dokumentversion: \curVers{}\versiontext}}

%% Acceptance Criterions
\newenvironment{acceptance}[1][hpiblue4]{%
\begin{center}	
	\tablefirsthead{%
		\hline
		\multicolumn{2}{|l|}{\cellcolor{hpiblue3}\bfseries Abnahmekriterien:}\\
		\hline}
	\tablehead{%
		\hline
		\multicolumn{2}{|l|}{\cellcolor{hpiblue3}\bfseries Abnahmekriterien (Fortsetzung):}\\
		\hline}
	\tabletail{%
		\multicolumn{2}{|r|}{\cellcolor{hpiblue4}\small\sl Fortsetzung auf der n�chsten Seite}\\
		\hline}
	\tablelasttail{}
	\begin{supertabular}{p{.23\linewidth}p{.7\linewidth}}
	}{%
	\end{supertabular}
\end{center}
\vfill
}

\newcommand{\criterion}[3]{%
	& \\
	\rowcolor{hpiblue4}Ausgangssitiation: & #1 \\
	Ereignis: & #2 \\
	Erwartetes Ergebnis: & #3 \\
}

\newcommand{\authindex}[1]{\expandafter\index{#1}}
%% SecAuthor
\newcommand{\secauthor}[2]{%
\def\secChap{chapter}
\def\secSect{section}
\def\secSubs{subsection}
\edef\refer{#1!Abschnitt \thesection}
\def\sec{#2}
\ifx\sec\secChap\edef\refer{#1!Kapitel \thechapter}\fi
\ifx\sec\secSect\edef\refer{#1!Abschnitt \thesection}\fi
\ifx\sec\secSubs\edef\refer{#1!Abschnitt \thesubsection}\fi
\label{auth:\theauthcounter}
\authindex{\refer}
%In Datei schreiben
%\WriteAuthor{#2}
\stepcounter{authcounter}
}

\newif\ifnotdone

\newcommand{\readLine}[1]{%
\ifeof#1
	\def\tobedone{}
	\notdonefalse
\else
	\read\TodoFileIn to \tobedone
	\notdonetrue
\fi
\tobedone\par}

%%% XML-Command
\newdimen\LineFeedDim
\LineFeedDim = 1.5em
\newdimen\LineFeed
\newif\ifXMLintern

\newcommand{\Tag}[4][black]{%
\ifXMLintern\\\hskip\LineFeed\fi%
\XMLinternfalse%
\textcolor{#1}{<#2}%
\def\paratest{#3}%
\ifx\paratest\empty{}%
\else{} #3%
\fi%
\global\advance\LineFeed by \LineFeedDim%
\def\contenttest{#4}%
\ifx\contenttest\empty%
	\global\advance\LineFeed by -\LineFeedDim\textcolor{#1}{/>}%
\else%
\textcolor{#1}{>}\\\hskip\LineFeed#4\\%
\global\advance\LineFeed by -\LineFeedDim\ifdim\LineFeed > 0em\hskip\LineFeed\fi\textcolor{#1}{</#2>}%
\fi\XMLinterntrue%
}

%% <? ... ?> als Argument �bergeben -> processing, comment, normal
\def\proctest{processing}
\def\commtest{comment}
\def\normtest{normal}
\newcommand{\LineTag}[4][normal]{%
\ifXMLintern\\\hskip\LineFeed\fi%
\XMLinternfalse%
\def\argtest{#1}%
<\ifx\argtest\proctest ?\else\ifx\argtest\commtest !-- \fi\fi#2%
\def\partest{#3}%
\ifx\partest\empty%
\else{} %
	#3%
\fi%
\def\contenttest{#4}%
\ifx\contenttest\empty%
\def\argtest{#1}%
\ifx\argtest\proctest{} ?\else\ifx\argtest\commtest{} --\else/\fi\fi>%
\else> #4 </#2>\fi\XMLinterntrue%
}

\newcommand{\EmptyTag}[1][]{%
\ifXMLintern\\\hskip\LineFeed\fi#1\parbox[c][1ex][c]{1ex}{}\XMLinterntrue%
}

\newcommand{\NewLinePar}{%
\\\hskip\LineFeed\hskip3em
}

\newcommand{\xml}[3][black]{%
	\LineFeed=0em
	\XMLinternfalse
	\small
	\fcolorbox{#1}{#2}{\ttfamily\shortstack[l]{#3}}
}

\newcommand{\soapmsg}[7][hpigray4]{%
	{\centering
	\begin{tabularx}{\linewidth}{|l|X|}
		\hline
		\cellcolor{#1}K�rzel & #2 \\
		\hline
		\cellcolor{#1}Consumer & #3 \\
		\hline
		\cellcolor{#1}Request Parameter & #4 \\
		\hline
		\cellcolor{#1}Response Parameter & #5 \\
		\hline
		\cellcolor{#1}Kurzbeschreibung & #6 \\
		\hline%
		\cellcolor{#1}Doppelter Request & #7 \\
		\hline
	\end{tabularx}
	}
}

\newcommand{\myabstract}[2]{%
	\def\germtest{#1}
	\def\engltest{#2}
	\ifx\germtest\empty
		\ifx\engltest\empty
		\else
			\hbox{ }
			\vfill
		\fi
	\else
		%\hbox{ }
		%\vfill
  \fi
	\ifx\germtest\empty\else
  	\begin{quotation}
  	\begin{center}\normalfont\sectfont\nobreak Kurzfassung\end{center}
  	#1
  	\end{quotation}
  	\vskip1cm
  	\clearpage
  \fi
  \ifx\engltest\empty\else
  	\begin{quotation}
  	\begin{center}\normalfont\sectfont\nobreak Abstract\end{center}
  	#2
  	\end{quotation}
	\fi
	\ifx\germtest\empty
		\ifx\engltest\empty
		\else
  		\vfill
  		\vfill
  		\clearpage
		\fi
	\else
  	\vfill
  	\vfill
  	\clearpage
  \fi
}